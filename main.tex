\documentclass[11pt]{article}
\usepackage{amsfonts}
\usepackage{amsmath}
\usepackage{amsthm}
\usepackage{amssymb}
\usepackage{mathrsfs}
\usepackage[numbers]{natbib}
\usepackage[fit]{truncate}
\usepackage{fullpage}

\newcommand{\truncateit}[1]{\truncate{0.8\textwidth}{#1}}
\newcommand{\scititle}[1]{\title[\truncateit{#1}]{#1}}

\theoremstyle{plain}
\newtheorem{theorem}{Theorem}[section]
\newtheorem{corollary}[theorem]{Corollary}
\newtheorem{lemma}[theorem]{Lemma}
\newtheorem{claim}[theorem]{Claim}
\newtheorem{proposition}[theorem]{Proposition}
\newtheorem{question}{Question}
\newtheorem{conjecture}[theorem]{Conjecture}
\theoremstyle{definition}
\newtheorem{definition}[theorem]{Definition}
\newtheorem{example}[theorem]{Example}
\newtheorem{notation}[theorem]{Notation}
\newtheorem{exercise}[theorem]{Exercise}

\begin{document}

\title{A Game of Combinations --- the Imtiaz - Wajid Genotype Formula}
\author{Aitzaz Imtiaz}
\date{}
\maketitle


\begin{abstract}
 In this paper, we will be studying the biological concept of Heredity, with its Mathematical blend. In short, we explain that:
 $$
 {}^{n}G_{2}={}^{2n}C_{2}-n
 $$
 is the Genotype Formula, a complete alternative to the Test Cross, to determine the total count of Genotypes of offspring, but returning no answer for the ratio of such result.
\end{abstract}
\section{Introduction}
The first time the results were considered, it was a normal 2022 lecture \cite{1} when I first came to know about the idea of Test Cross. To pass down the comment, the idea of both Punnet's Square and the Test Cross remains completely un-reliable to this date. 

The basic idea of Punnet's square was to give the result of multiple alleles in the form of $(n-1)^2$.  Here, $n$ is a square with the most northwest part of the square as $\phi$. So, 3 would be the combination of two parental Genotypes. The following is not only hard but an insecure calculation of probability. The Imtiaz Wajid Genotype Function solves the such problem.
\section{Formulation}
We carry on to simplify $ {}^{n}G_{2}={}^{n}C_{2}-n$ for an ordinary Biologist to understand how the flow of such a model works.
$$
{}^{n}G_{2}={}^{2n}C_{2}-n
$$
$$
{}^{2n}C_{2}-n = \frac{(2n)!}{(2)!(2n-2)!}-n
$$
$$
=\frac{2n(2n-1)}{2}-n
$$
$$
=2n^2-2n
$$
You can refer back to the front page results of the book\cite{1} to see, the main results, when 3 parents make a test cross of heredity, the answer is 12, and when you apply the same 3 in the equation we make, the answer will be 12.
\section{Combinatorial Relation}
The following is the mathematical affiliation of the results derived in the formulation. We can carry on the results further by the extension as follows:
\newline
$$
{}^{n}G_{2}={}^{2n}C_{2}-n=2n^2-2n
$$
$$
{}^{2n}C_{2}-n=2(n^2-n)
$$
$$
{}^{2n}C_{2}-n=2{}^{n}P_{2}
$$
$$
{}^{2n}C_{2}=2{}^{n}P_{2}+n
$$
$$
{}^{n}G_{2}=2{}^{n}P_{2}
$$
\section{Conclusion}
We end up making a new formula in the form of ${}^{n}G_{2}$ which helps in knowing the Test Cross total Genotype results over large datasets.
\begin{footnotesize}
\bibliography{bibfile}
\bibliographystyle{plainnat}
\end{footnotesize}

\end{document}








